% SolSeed Creed Commentary
% (c) by Brandon CS Sanders, Shelley Sanders, Benjamin Sibelman,
% Eric Saumur, and other contributing members of The SolSeed Movement
%
%
% SolSeed Creed Commentary is licensed under a
% Creative Commons Attribution-ShareAlike 3.0 Unported License.
%
% You should have received a copy of the license along with this
% work.  If not, see <http://creativecommons.org/licenses/by-sa/3.0/>.
\documentclass[ebook,12pt,openany,twoside]{memoir}
\usepackage[latin1]{inputenc}
\usepackage{setspace}
\usepackage{tocloft}
\usepackage{graphicx}
\usepackage{eso-pic}
\chapterstyle{bringhurst}
\openright

\setlength\stockheight {9.25in}% \stockheight=9.25in
\setlength\stockwidth  {6.25in}% \stockwidth=6.25in
\setlength\trimtop     {.125in}
\setlength\trimedge    {.125in}
\setlength{\paperwidth}{6.0in}
\setlength{\textwidth}{4.6in}
\setlength{\textheight}{6.5in}
\epigraphposition{flushright}
% \setlength\epigraphwidth{3.6in}

\newcommand{\tab}{\hspace*{2em}}

\newcommand{\imagefacingchapter}[1]{
  \cleartoverso
  \clearpage \null
  \thispagestyle{cleared}
  \AddToShipoutPictureBG*{% Add picture to current page
    \AtStockLowerLeft{% Add picture to lower-left corner of paper stock
      \includegraphics[keepaspectratio=true, height=\stockheight]{#1}
    }
  }
  \clearpage
}

\begin{document}


%\pagenumbering{}
% Set book title
\title{\textbf{SolSeed Creed Commentary}}
% Include Author name and Copyright holder name
\author{The SolSeed Movement}
\begin{titlingpage}
\maketitle
\end{titlingpage}


\cleartorecto
\thispagestyle{cleared}
\pagenumbering{arabic}

\pagestyle{plain}

\imagefacingchapter{images/LifeIsWorthyOfVeneration}
\chapter{Life is Precious}

\setlength\epigraphwidth{2.25in}
\epigraph{
  Life is precious.\\
  \tab It has always been precious,\\
  \tab it will always be precious.
}{}

\noindent Life puts atoms together in very interesting ways. One configuration
of atoms gives us an ant carrying a leaf many times larger than itself. From
another configuration we get an eagle slipping gracefully through the sky. Yet
another configuration becomes a person who feels, thinks, and loves.

Without Life, the energy from our sun simply bounces off the planet. Life
literally collects the energy Sol sends to earth, and stores it up for future
use. Plants take energy from Sol and convert it into denser, more storable
forms. The sugars and fats that our bodies burn are little pools of energy that
originally came from our sun.

Ants carrying, eagles soaring, people loving \ldots none of these would exist
without Life's awesome capacity for organizing matter.

Life is not perfect. This is obvious to anyone who has experienced cruelty or
misfortune. And yet, Life is the only game in town. Without Life the Earth
would be just another dead ball of rock.

Nothing needs to change for Life to become ``worthy''. It already is worthy.
Even if we had more wars, more hunger, and more systemic injustice, Life would
still still be precious. Always and forever, Life is what it is \ldots and it
is worthy of veneration!

\imagefacingchapter{images/UpwardSpiral}
\chapter{The Upward Spiral of Life}

\setlength\epigraphwidth{3.6in}
\epigraph{
  Life creates the conditions for more Life,\\
  \tab in an Upward Spiral of ever-growing possibilities.\\
}{}

\noindent Every one of us knows from personal experience that it is easier to
destroy things than it is to create them. In fact, we don't even need to exert
ourselves to destroy things. If we just ignore them for a while they will erode
and disappear. The destruction of possibility is the natural order of things.
Scientists call this tendency for things to run down the ``second law of
thermodynamics.'' The second law can be summarized as ``energy always flows
downhill toward less and less usable forms.''

And yet many wondrously complex things do exist! In our world that is strongly
biased towards decay and collapse, how is this possible?

Life is the force that creates an upward eddy in the downward current of
destruction. Creation by life is a process of slow growth. The laying down of
one layer on top of another. This slow process of creation often spans
generations.

Turning to Paul Krafel's ``The Upward Spiral'', we find succession as more than
just a brutal competition in which taller plants shade out shorter plants. In
fact, the shorter plants that came first created the conditions that allowed
the taller plants to exist at all. The story is about competition, but it is
not just about competition. It is also a story of multi-generational,
multi-species cooperation to slowly raise up greater and greater possibilities.

The process of evolution that created the wonderful diversity of species is a
long slow steady climb. And like the story of succession, it is a story of
competition AND a story of cooperation. The long slow steady climb of evolution
is the story of the creation of possibilities by countless individuals making
countless contributions across countless generations.

Ants carrying, eagles soaring, people loving \ldots none of these would exist
without Life's slow and steady creation. Life creates an upward spiral of
ever-growing possibilities!

\imagefacingchapter{images/BodyOfAllLife-cropped}
\chapter{SolSeed: The Body of All Life}

\setlength\epigraphwidth{2.4in}
\epigraph{
  As you are alive, and I am alive,\\
  and in kinship with all other beings\\
  nourished by Sol,\\
  \tab we are SolSeed,--- \\
  \tab the body of all Life.
}{}

\noindent Groups of people may be organized into bodies. When we regard a
number of individuals as a single entity, we refer to the group as a ``body''.
We have legislative bodies, student bodies, administrative bodies, governing
bodies, and church bodies. Bodies are in some way more than the sum of the
individuals. We expect a body of people to generate ideas and think about
things in a way that is different from what the individuals alone would come up
with.

Groups of cells may also be organized into bodies.  The cells in your body don't
know who you are and don't care about you. And yet, by each cell doing its own
little thing in its own little context, this miraculous thing called
you emerges!

Just like our body is composed of cells that each do their own different unique
thing, so too there is a body of all life composed of living organisms of which
we are a part. This body of all Life has many names. We call it SolSeed. Sol
out of respect for the star at the center of our system that provides
practically all the energy that animates Life on Earth, and Seed to focus our
attention on the myriad possibilities latent in the complex biosystem Life has
created by storing the gift of energy that comes from Sol.

\imagefacingchapter{images/MotherEarthFatherSun}
\chapter{Life Takes Root, Flowers, and Spreads}

\setlength\epigraphwidth{3.0in}
\epigraph{
  The Destiny of SolSeed\\
  is to take root amongst the stars ---\\
  \tab to give birth to a family of living worlds.
}{}

\noindent It is the nature of Life to flower and spread. During the Devonian
some 420 million years ago, Life burst out of the sea and took root on the
land. Life greatly diversified and began to substantially affect the landscape.
The changes Life made to the landscape created opportunities for even more
Life. This upward spiral of adaptive radiation was a period of intense
innovation leading to more and more possibilities for Life.

The evolution of Life in our solar system is far from complete. Earth Life is
now poised to burst off of our planet and take root amongst the other bodies of
our Solar system. The adaptive radiation that follows will again be a period of
tremendous innovation for Life. Life will begin to substantially affect the
rest of the Solar system. The changes Life will make will create opportunities
for even more Life.

Mother Earth and Father Sun are ready to start a family. Instead of our solar
system containing a single, lonely living world, there will be a family of
hundreds, thousands, and eventually millions of living worlds. Life will have a
tremendously enlarged playground over which to diversify.

In the very far future, Solar Life will spread beyond this solar system to take
root amongst the stars. As it does so, it may encounter Life from other sources
with which to commune.  Just imagine the possibilities!


% Last pages for ToC
%-------------------------------------------------------------------------------
\newpage
% Include dots between chapter name and page number
%Finally, include the ToC
\tableofcontents




\end{document}
